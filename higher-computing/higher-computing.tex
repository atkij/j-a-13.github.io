\documentclass[a4paper, 12pt]{report}
\usepackage{fullpage}
\usepackage{graphicx}
\usepackage{amsmath}

\setlength{\parindent}{0pt}

\author{Joel Atkinson}
\title{Higher Computing Notes}
\date{}

\begin{document}

\maketitle

\tableofcontents

\chapter{Computer Systems}

\section{Data Representation}

\subsection{Two's Complement}

\subsubsection{Decimal to Binary}

To create the binary of -120, there are 3 steps.

\begin{enumerate}
\item Convert the positive number into binary

0111 1000 (120)

\item Invert all the bits (0's to 1's and 1's to 0's)
	
1000 0111

\item Add 1 to the right of the number (Least Significant Bit)

\begin{tabular}{r}
1000 0111 \\
+ 0000 0001 \\
\hline
1000 1000 \\
\end{tabular}
\end{enumerate}

\subsubsection{Binary to Decimal}

To convert a tow's complement number from binary to decimal, simply write out the column headers except with the most significant bit as negative, then add up all the numbers with a 1. \\

\begin{tabular}{ccccccccl}
-128 & 64 & 32 & 16 & 8 & 4 & 2 & 1 & \\
1 & 0 & 0 & 0 & 1 & 0 & 0 & 0 & = -128 + 32 = -120\\
\end{tabular}

\subsubsection{Min and Max Limits}

To calculate the maximum and minimum numbers that can be stored using the two's complement system, use the following formulae: \\

From $-2^{n-1}$ to $(2^{n-1})-1$

\subsection{Floating Point Representation}

\subsubsection{Binary Fractions}

To store binary fractions, we simply add extra column headers to the right of the decimal point by dividng by 2. \\

\begin{tabular}{cccccccccccc}
128 & 64 & 32 & 16 & 8 & 4 & 2 & 1 & . & 0.5 & 0.25 & 0.125 \\
$2^{7}$ & $2^{6}$ & $2^{5}$ & $2^{4}$ & $2^{3}$ & $2^{2}$ & $2^{1}$ & $2^{0}$ & . & $2^{-1}$ & $2^{-2}$ & $2^{-3}$ \\
\end{tabular}\\

To convert a fixed point binary to decimal, we just add up the column headers as before. \\

Because the computer can't store decimal points in binary, we have to use floating point representation to store real numbers.

\subsubsection{Storing Floating Point Numbers}

A floating point number is split up into 3 parts - the sign, the mantissa and the exponent.  The sign determines whether it is a negative number or not (1 for a negative and 0 for positive just as in two's complement).  The mantissa is the whole number without the decimal point.  This is left-aligned and padded with 0's.  The exponent is used to tell the computer the position of the decimal point in the mantissa.  This right aligned and padded with 0's at the start. \\

\begin{tabular}{cccc}
& Sign & Mantissa & Exponent \\
24 bits = & 1 bit & 15 bits & 8 bits \\
32 bits = & 1 bit & 23 bits & 8 bits \\
64 bits = & 1 bit & 52 bits & 11 bits \\
\end{tabular}

\subsubsection{Converting Real Numbers to Fixed Point Binary}

To convert a real number like 250.03125 to fixed point binary, we must: \\

\begin{enumerate}
\item Convert the first part of the number into binary (before the decimal point)

250.03125 \\
250 = 111 1010

\item Convert the second part of the number into binary by multiplying until it equals 1

$\left.
\begin{tabular}{ll}
0.03125 & * 2 = 0r0.0625 \\
0.0625 & * 2 = 0r0.125 \\
0.125 & * 2 = 0r0.25 \\
0.25 & * 2 = 0r0.5 \\
0.5 & * 2 = 1r0 \\
\end{tabular}
\right\downarrow
= 0.00001$

\item Now join the two parts together to get:

1111010.00001
\end{enumerate}

\subsubsection{Converting Fixed Point Binary to Floating Point}

We start with a fixed point binary number such as 11.001011 to convert to 24 bit floating point representations:  \\

\begin{enumerate}
\item We first move the decimal point out of the number to the left and then pad it out on the right to form the mantissa.

11.001011 \\
110010110000000

\item We then convert the number of spaces the dcimal point moved to form the exponent

It moved 2 spaces so the exponent is 0000 0010

\item We then choose the sign and fill out the table

\begin{tabular}{ccc}
Sign & Mantissa & Exponent \\
0 & 110 0101 1000 0000 & 0000 0010 \\
\end{tabular}
\end{enumerate}

If the number is a smaller number (below 0), then the decimal point must be moved to the right so that is just to the left of the firs 1.  Because the decimal point has been moved in the opposite direction, then the exponent will be a negative number and must be stored using the two's complement system.

\subsubsection{Bit Allocation}

The higher the number of bits stored in the mantissa, the higher the precision of the number.  The more numbers after the decimal point, the greater the precision. \\

Increasing the exponent increases the range of numbers that can be stored as the decimal point can be moved more places to the right and left. \\

Single precision is a 32 bit floating point number (1 bit sign, 23 bits mantissa, 8 bits exponent) and double precision is  64 bits (1 bit sign, 52 bits mantissa, 11 bits exponent).

\subsection{Storing Text}

\subsubsection{ASCII}

ASCII is a method of storing text digitally.  Each letter on a keyboard has it's own ASCII code which was originally stored with 7 bits but was increased to 8 bits to store more characters for more languages and symbols (known as extended ASCII).  ASCII stands for the American Standard Code for Information Interchange. \\

The first 32 codes are known as control codes which cannot be printed such as backspace, delete and escape. \\

A character set is the term for a certain way of storing text on a computer.  There are lots of different character sets for different languages.  ASCII is one character set and Unicode is another.

\subsubsection{Unicode}

The problem with ASCII is that there wasn't enough space to store every letter of every language's alphabet.  Unicode was invented to fix this by using more bits so it could store more letters and symbols. \\

Unicode is known as UTF-8, UTF-16 and UTF-32, storing 8, 16, 32 bits for each character. \\

The first 128 values of Unicode are the same as ASCII and then goes up to 137,439 characters!

\subsection{Graphics}

\subsubsection{Vector Graphics}

\begin{description}
\item[Advantages]:
\begin{itemize}
\item Do no lose quality when scaled
\item Require less sotrage space
\item Objects are easily moved/manipulated
\item Resolution independent
\end{itemize}

\item[Disadvantages]:
\begin{itemize}
\item Cannot be edited at pixel level
\item Cannot show photo realistic scenes easily
\item Will usually require special applications to open
\end{itemize}
\end{description}

\subsubsection{Bitmapped Graphics}

\begin{description}
\item[Advantages]:
\begin{itemize}
\item Can edit down to the individual pixel
\item Adding extra detail to the picture does not increase the file size
\item Can create photo realistic images
\end{itemize}

\item[Disadvantages]:
\begin{itemize}
\item Have a very large file size as every pixel is stored
\item When scaled larger, pictures pixelate as they are resolution dependent
\end{itemize}
\end{description}

\chapter{Software Design and Development}
Hello world

\end{document}
